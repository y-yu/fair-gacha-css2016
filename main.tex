\definecolor{links}{HTML}{2A1B81}
\hypersetup{colorlinks,linkcolor=,urlcolor=links}

\usetheme{Boadilla}
\usecolortheme{seahorse}
\usefonttheme{serif}
\beamertemplatenavigationsymbolsempty

\setbeamertemplate{bibliography item}{\insertbiblabel}
\setbeamersize{description width=1cm}

\usepackage{luacode}
\usepackage{luatexja}
\usepackage{pgfpages}
\usepackage[osf]{mathpazo}

\begin{luacode*}
  USE_IPAFONT = os.getenv"USE_IPAFONT"
  USE_YUFONT = os.getenv"USE_YUFONT"
  
  if USE_YUFONT == "true" then
    tex.sprint("\\AtBeginDocument{\\usepackage[yu-osx, deluxe, expert]{luatexja-preset}}")
  elseif USE_IPAFONT == "true" then
    tex.sprint("\\AtBeginDocument{\\usepackage[ipaex, deluxe, expert]{luatexja-preset}}")
  else
    tex.sprint("\\AtBeginDocument{\\usepackage[hiragino-pro, deluxe, expert]{luatexja-preset}}")
  end
\end{luacode*}

\usepackage{epigraph}
\usepackage{etoolbox}
\usepackage{tikz}
\usepackage{framed}
\usepackage{libertine}
\usepackage{amsmath}
\usepackage{mathtools}

\renewcommand{\kanjifamilydefault}{\gtdefault}

%\setbeameroption{show notes on second screen=right}

\setmainfont[Numbers=OldStyle, BoldFont=Palatino Bold]{Palatino}
\setsansfont{CMU Sans Serif}
\setmonofont{CMU Typewriter Text}

\input{vc.tex}

\title[公平なガチャ]{%
  公平なガチャ \\
  {\normalsize \href{http://www.iwsec.org/css/2016/css2.htm}{CSS2016 CSS2.0}}
}
\author{吉村 優}
\date[October 11, 2016]{%
  October 11, 2016 \\%
  {\footnotesize (Commit ID: \GITAbrHash)}
}
\institute[株式会社ドワンゴ]{%
  株式会社ドワンゴ \\
  \href{mailto:hikaru\_yoshimura@dwango.co.jp}{hikaru\_yoshimura@dwango.co.jp}

}

\input{./lib/quotebox.tex}
\input{./lib/footnotemark.tex}
\input{./lib/ballon.tex}

\newcommand\ballref[1]{%
\tikz \node[circle, shade,ball color=structure.fg,inner sep=0pt,%
  text width=8pt,font=\tiny,align=center] {\color{white}\ref{#1}};
}

\begin{document}

\frame{\maketitle}

\section{自己紹介}
\begin{frame}
  \frametitle{自己紹介}
  
  \begin{columns}
    \begin{column}{0.4\textwidth}
      \centering
      \begin{figure}
        \includegraphics[width=0.95\textwidth]{img/bird2x.png}
      \end{figure}

      \begin{description}
        \item[Twitter] \href{https://twitter.com/\_yyu\_}{@\_yyu\_}
        \item[Qiita] \href{http://qiita.com/yyu}{yyu}
        \item[GitHub] \href{https://github.com/y-yu}{y-yu}
      \end{description}
    \end{column}
    \begin{column}{0.6\textwidth}
      \begin{itemize}
        \item<2-> 筑波大学 情報科学類卒(学士)
        \item<3-> 株式会社ドワンゴ 入社
        \item<4-> 基盤開発本部 共通基盤開発部 \\
          認証基盤セクション
      \end{itemize}
    \end{column}
  \end{columns}
\end{frame}

\section{ガチャとは?}

\begin{frame}
  \frametitle{ガチャとは?}

  \begin{itemize}
    \item<2-> ソーシャルゲームなどに実装された機能のひとつ
    \item<3-> ユーザーは、お金を投入すると\textbf{確率}で特定の景品を入手できる
  \end{itemize}

  \begin{center}
    \uncover<4->{
      \begin{tikzpicture}
        \calloutquote[width=5cm,position={(0.7,-0.2)},fill=blue!30,rounded corners]{誰が確率を計算するの?}
      \end{tikzpicture}
    }

    \uncover<5->{
      \begin{tikzpicture}
        \calloutquote[width=4cm,position={(-0.5,-0.2)},fill=green!30,rounded corners]{運営のサーバー?}
      \end{tikzpicture}
    }

    \uncover<6->{
      \begin{tikzpicture}
        \calloutquote[width=9cm,position={(0.9,-0.2)},fill=red!30,rounded corners]{サーバーの実装が正しい確率に従っているのか?}
      \end{tikzpicture}
    }
  \end{center}
\end{frame}

\begin{frame}
  \frametitle{ガチャの確率に対する疑惑}

  サーバーで計算される確率に対して疑惑を抱いている人もいる

  \begin{center}
    \uncover<2->{
      \begin{tikzpicture}
        \calloutquote[width=11cm,position={(-0.7,-0.2)},fill=green!30,rounded corners]{サーバーで計算される確率をユーザーが検証するのは無理}
      \end{tikzpicture}
    }

    \uncover<3->{
      \begin{tikzpicture}
        \calloutquote[width=7cm,position={(0.7,-0.2)},fill=blue!30,rounded corners]{クライアントサイドで計算すれば?}
      \end{tikzpicture}
    }

    \uncover<4->{
      \begin{tikzpicture}
        \calloutquote[width=7cm,position={(-0.7,-0.2)},fill=red!30,rounded corners]{リバースエンジニアリングの餌食!}
      \end{tikzpicture}
    }
  \end{center}
\end{frame}

\section{公平なガチャ}
\begin{frame}
  \frametitle{公平なガチャ}

  次のような\textbf{公平なガチャ}が欲しい
  \begin{itemize}
    \item<2-> ユーザーにとっても運営にとっても、ガチャによる景品の出現確率が実装に基づいて明らかである
    \item<3-> 悪意を持つユーザーや、悪意を持つ運営による確率操作ができない
  \end{itemize}
\end{frame}

\section{コミットメント}

\begin{frame}
  \frametitle{コミットメント}

  公平なガチャを実装するために\textbf{コミットメント}を使う

  \uncover<2->{
    \begin{block}{コミットメント}
      \begin{description}
        \item[コミット]<3->\mbox{}\\
          送信者はコミットしたい情報$b$を暗号化して受信者に送信する
        \item[公開]<4->\mbox{}\\
          送信者は受信者が$b$を復元できるように付加的な情報$r$を受信者に送信する 
      \end{description}
    \end{block}
  }

  \uncover<5->{
    \begin{exampleblock}{コミットメントの性質}
      \begin{description}
        \item[隠蔽] コミットのステップでは、受信者はコミットされた値について何も分からない
        \item[束縛] 送信者はコミットのステップ後に、コミットした値を変更することができない
      \end{description}
    \end{exampleblock}
  }
\end{frame}


\section{コミットメントを用いたガチャ}

\begin{frame}
  \frametitle{コミットメントを用いたガチャ}
  
  \begin{enumerate}
    \item<2-> \label{item:assign}
      運営は、$N$種類の景品$K_1 \dots K_N$に番号を割り当てる
      \begin{table}[h]
        \begin{tabular}{c|c}
          景品 & 番号 \\
          \hline \hline
          $K_1$ & $1$ \\
          $\vdots$ & $\vdots$ \\
          $K_N$ & $N$ \\
        \end{tabular}
      \end{table}
    \item<3-> 運営は割り当てた番号をユーザーへ公開する
    \item<4-> 運営は\textbf{キー}と呼ばれる数値$m \in \{1 \dots N\}$を用意する
    \item<5-> 運営はキーのコミットメント$c := C(r, m)$を計算し、$c$をユーザーへ送信する
    \item<6-> ユーザーはコミットメント$c$を受けとり、
      $x \in \{1 \dots N\}$を選択し、$x$を運営へ送信する
  \end{enumerate}
\end{frame}

\begin{frame}
  \frametitle{コミットメントを用いたガチャ}

  \begin{enumerate}
    \setcounter{enumi}{5}
    \item<1-> 運営は$m$と$r$を公開する
    \item<2-> ユーザーは$c = C(r, m)$を検証する
    \item<3-> ユーザーは次の$i$に対応する景品$K_i$を得る
      \[
       i := (x + m) \bmod N
     \]
  \end{enumerate}
\end{frame}

\section{隠蔽と束縛とガチャ}

\begin{frame}
  \frametitle{隠蔽と束縛とガチャ}

  \begin{center}
    \uncover<2->{
      \begin{tikzpicture}
        \calloutquote[width=8cm,position={(-0.7,-0.2)},fill=red!30,rounded corners]{隠蔽と束縛を同時に満すことはできない\footnote[frame]{両方を満すコミットメントを仮定する背理法で証明できる}!}
      \end{tikzpicture}
    }

    \uncover<3->{
      \begin{tikzpicture}
        \calloutquote[width=7cm,position={(0.7,-0.2)},fill=green!30,rounded corners]{どちらかを\textbf{計算の複雑さ}に依存させる}
      \end{tikzpicture}
    }

    \uncover<4->{
      \begin{tikzpicture}
        \calloutquote[width=9cm,position={(-0.7,-0.2)},fill=cyan!30,rounded corners]{隠蔽と束縛の不完全さがガチャに影響を与える?}
      \end{tikzpicture}
    }
  \end{center}
\end{frame}


\begin{frame}
  \frametitle{隠蔽と束縛とガチャ}

  \setlength{\leftmarginii}{0pt}
  \begin{description}
    \item[隠蔽が計算の複雑さに依存する場合]\mbox{}\\
      \begin{itemize}
        \item<2-> ユーザーは計算によってコミットされた情報から平文を復元できる
        \item<3-> 平文をオークションなどで売買するようになる
          \begin{center}
            \uncover<4->{
              \begin{tikzpicture}
                \calloutquote[width=6cm,position={(-0.7,-0.2)},fill=green!30,rounded corners]{\textbf{ガチャ採掘師}が誕生する?}
              \end{tikzpicture}
            }
          \end{center}
        \item<5-> 景品に市場価格が生れる可能性がある
      \end{itemize}
    \item[束縛が計算の複雑さに依存する場合]\mbox{}\\
      \begin{itemize}
        \item<6-> 一般に運営はユーザーより豊富な計算資源を持っている
        \item<7-> 運営が景品を意図的に操作していると思われる
          \begin{center}
            \uncover<8->{
              \begin{tikzpicture}
                \calloutquote[width=9cm,position={(0.7,-0.2)},fill=cyan!30,rounded corners]{公平なガチャが公平ではないと思われてしまう?}
              \end{tikzpicture}
            }
          \end{center}
      \end{itemize}
  \end{description}
\end{frame}

\section{ガチャと署名}

\begin{frame}
  \frametitle{悪意あるユーザーと悪意ある運営}
  
  \begin{description}
    \item[悪意あるユーザー]<2->\mbox{}\\
      自分でコミットを適当に作成し、それを運営が作成したと嘘の主張する
    \item[悪意ある運営]<3->\mbox{}\\
      自らが作成したコミットを、ユーザーが作成した嘘のコミットと主張する
  \end{description}

  \begin{center}
    \uncover<4->{
      \begin{tikzpicture}
        \calloutquote[width=8cm,position={(-0.7,-0.2)},fill=green!30,rounded corners]{これらの不正を排除するために\textbf{署名}を使う}
      \end{tikzpicture}
    }

    \uncover<5->{
      \begin{tikzpicture}
        \calloutquote[width=9cm,position={(0.7,-0.2)},fill=cyan!30,rounded corners]{運営はコミットに署名をして偽造と否認を拒否}
      \end{tikzpicture}
    }

    \uncover<6->{
      \begin{tikzpicture}
        \calloutquote[width=7cm,position={(-0.7,-0.2)},fill=red!30,rounded corners]{検証鍵を配布する機関が必要になる}
      \end{tikzpicture}
    }
  \end{center}
\end{frame}

\section{まとめ}
\begin{frame}
  \frametitle{まとめ}
  
  \begin{itemize}
    \item<2-> ガチャにおけるサーバーサイドの確率計算に疑念を抱くユーザーがいる
    \item<3-> 今回提案したガチャは、コミットメントを用いてユーザーが確率を検証できるようにした
    \item<4-> コミットメントにおける隠蔽と束縛がガチャに大きな影響を与える
    \item<5-> 電子署名を用いてコミットを偽造したり否認したりを防ぐようにした
  \end{itemize}
\end{frame}

\section*{参考文献}
\begin{frame}
  \frametitle{参考文献}

  \bibliographystyle{junsrt_url}
  \nocite{*}
  \bibliography{ref}
\end{frame}

\begin{frame}
  \frametitle{目次}

  \tableofcontents[hideallsubsections]
\end{frame}

\begin{frame}
  \centering
  {\Huge Thank you for listening!}

  \quad \quad

  {\Huge Any question?}
\end{frame}

\end{document}
